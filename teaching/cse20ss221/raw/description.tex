\documentclass[12pt]{article}

% Some basic math packages
\usepackage{amsmath, amssymb}

% This is a macro
\newcommand{\intdiv}[2]{\left( #1 \ \textbf{div} \ #2 \right)}

\title{CSE 20 Project}
\date{Summer Session 2, 2021}
\author{}

\begin{document}
\maketitle
\section*{Description}

This project replaces small exams given in past summers.
This will be an individual assessment, designed to have you demonstrate
problem-solving skills \emph{and} to improve your technical writing.
Clear communication is an invaluable skill, and the project will give each
of you the chance to develop it.
The first two parts of the project will be written math problems. We will grade
them and give you feedback on your technical writing.
The third part of the project will allow
you to incorporate this feedback by revising a prior submission.
We will drop the lowest score when computing your final grade -- that means
your grade on the revised submission will replace the original grade.

In general, your focus should be on clarity. The reader should be able to follow
your reasoning without any doubt as to how a particular conclusion was reached.

All components are due at 11:59pm on the deadline.


\section*{Part 1: Due 8.11.21}

We define the following functions on pairs of $3$-tuples:

\begin{align*}
  d_1((x_1, x_2, x_3), (y_1, y_2, y_3)) &= \sum_{i = 1}^3 \intdiv{\left( \left| x_i - y_i \right| + 1 \right)}{2} \\
  d_2((x_1, x_2, x_3), (y_1, y_2, y_3)) &= \sqrt{ \sum_{i = 1}^3 \left( x_i - y_i \right)^2 } 
\end{align*}

These definitions can be generalized to pairs of $n$-tuples.
Your written submission should include the following information: 

\begin{enumerate}
  \item Pick 4-6 movies or TV shows. Give them numerical ratings on a scale from
    1-10.
  \item Construct an $n$-tuple where the first element is the rating of the
    first movie/show, the second element is the rating of the second movie/show,
    and so on. Specify $n$.
  \item Write down new functions $d_1'$ and $d_2'$ that compute the same
    quantities for $n$-tuples (for your choice of $n$).
  \item Calculate the result of applying $d_1'$ to your $n$-tuple and an
    $n$-tuple consisting of all zeros. Explain your work.
  \item Calculate the result of applying $d_2'$ to your $n$-tuple and an
    $n$-tuple consisting of all zeros. Explain your work.
\end{enumerate}

\paragraph{Requirements:}
\begin{itemize}
  \item Names of shows/movies are listed, and there are 4-6 of them
  \item The $n$-tuple is defined appropriately, $n$ is given correctly, and all
    notation makes sense.
  \item The calculation of $d_1$ is correct and explained clearly.
  \item The calculation of $d_2$ is correct and explained clearly.
\end{itemize}

\section*{Part 2: Due 8.18.21}

RNA strands are nonempty \emph{strings} from the alphabet $B = \{ \texttt{A}, \texttt{G},
\texttt{C}, \texttt{U} \}$.
Let $S$ denote the set of RNA strands.

We define the following functions on RNA strands:

\begin{align*}
  \texttt{mutate}(b_1 \ b_2 \ldots b_n, k, b) &= b_1 \ b_2 \ldots b_{k - 1} \ b \ b_{k + 1} \ldots b_n \\
  \texttt{insert}(b_1 \ b_2 \ldots b_n, k, b) &= 
  \begin{cases}
    b_1 \ b_2 \ldots b_n \ b & k > n \\
    b_1 \ b_2 \ldots b_{k - 1} \ b \ b_k \ldots b_n & k \leq n
  \end{cases} \\
  \texttt{delete}(b_1 \ b_2 \ldots b_n, k) &= 
  \begin{cases}
    b_1 \ b_2 \ldots b_n & k > n \\
    b_1 \ b_2 \ldots b_{k - 1} \ b_{k + 1} \ldots b_n & k \leq n
  \end{cases}
\end{align*}

In each function above, the input $b_1 \ b_2 \ldots b_n$ is an RNA strand of
length $n$, $k$ is a non-negative integer index, and $b \in B$ is a character.
\texttt{mutate}
modifies a strand by replacing the $k$th character, \texttt{insert} inserts a new
character at position $k$, and \texttt{delete} deletes the $k$th character.

Now, consider the following logical predicates:

\begin{align*}
  Mut(s_1, s_2) &= \exists k \in \mathbb{Z}^+ \ \exists b \in B. \ \texttt{mutate}(s_1, k, b) = s_2\\
  Ins(s_1, s_2) &= \exists k \in \mathbb{Z}^+ \ \exists b \in B. \ \texttt{insert}(s_1, k, b) = s_2 \\
  Del(s_1, s_2) &= \exists k \in \mathbb{Z}^+. \ \texttt{delete}(s_1, k) = s_2
\end{align*}

In this assignment, you will come up with a logical statement about RNA strands
and analyze it. Below is an example statement:

\[ \forall s_1 \in S \ \forall s_2 \in S. \ Ins(s_1, s_2) \implies Del(s_2, s_1)  \]

Less formally, this means that for all RNA strands $s_1$ and $s_2$, if $s_2$ can
be formed by inserting a character into $s_1$, then $s_1$ can be formed by
deleting a character from $s_2$.

To complete the assignment, do the following:

\begin{enumerate}
  \item Using appropriate symbolic notation, write a quantified statement 
    about RNA strands. 
    If you'd like, define extra predicates for use in your statement.
    For example, it might be useful to reason about the length or contents of
    the string.
    Your quantified statement should have the following features:
    \begin{enumerate}
      \item Nested quantifiers, including at least one existential quantifier
        and one universal quantifier.
      \item One negation and one binary operator (and, or, etc)
      \item Negations appear only on predicates (that is, no negation is outside
        a quantifier or on an expression involving logical connectives)
      \item Use at least two of $Mut$, $Ins$, $Del$
    \end{enumerate}
  \item Translate your statement to english.
  \item Negate the whole statement and rewrite it such that negations appear
    only on predicates (that is, no negation is outside
    a quantifier or on an expression involving logical connectives)
  \item Prove or disprove your statement
\end{enumerate}

As always, show your work when necessary and justify each step.

\paragraph{Requirements:}
\begin{itemize}
  \item Quantified statement is clearly stated, well-defined, syntactically
    correct, and meets all requirements.
  \item Translation to english is clear, correct, and complete.
  \item The negation of the statement is clear, well-defined, syntactically
    correct, and meets all requirements.
  \item The proof (or disproof) of the statement is clear, correct, and
    complete.
\end{itemize}


\section*{Part 3: Due 9.1.21}

For this part you will revise your submission for \emph{either}
part 1 or part 2 of the project. Taking into account the feedback we gave you,
submit an improved version. If you aced it the first time, don't do anything! 
Please inclue a brief summary of the changes you made -- either by describing
the changes separately, or by highlighting the changes in-text.

\paragraph{Requirements:}
This will be graded accrding to the original requirements for the part you revise.
Make sure it's clear to the reader what you revised.

\end{document}
