\documentclass[12pt]{article}

\usepackage{amsmath, amssymb, amsthm}

\title{Supplemental Lecture Notes}
\date{August 10, 2021}
\author{Tristan Knoth}

\begin{document}
\maketitle

This document contains (semi) formally written versions of some proofs done in lecture
on August 10.
It should serve as one example of how to write a proof.

\subsection*{Slide 30}

\textit{Claim:} If $n \in \mathbb{Z}$ is even, then so is $n^2$
\begin{proof}
Let $n \in \mathbb{Z}$ be arbitrary.
Suppose $n$ is even.
By definition, there exists $k \in \mathbb{Z}$ such that
$n = 2k$.
Thus, $n^2 = 4k^2 = 2(2k^2)$.
Since the integers are closed under multiplication, $2k^2$ is an integer.
Thus, we know that $n^2$ is even by definition.
\end{proof}

\subsection*{Slide 31}

\textit{Claim:} If $n \in \mathbb{Z}$ is odd, then so is $n^2$
\begin{proof}
Let $n \in \mathbb{Z}$ be arbitrary.
Suppose $n$ is odd. 
Then by definition there exists an integer $k$ such that $n = 2k + 1$.
Thus, we have 
\begin{align*}
  n^2 &= (2k + 1)^2 \\
  &= 2k^2 + 4k + 1 \\
  &= 2(k^2 + 4k) + 1
\end{align*}
Since the integers are closed under multiplication and addition,
$k^2 + 4k$ is also an integer and $n^2$ is therefore odd by definition.
\end{proof}

\subsection*{Slide 35}

\textit{Claim:} If $x^2 - 6x + 5$ is even, then $x$ is odd.
\begin{proof}
We consider instead the contrapositive: for all integers $x$, if $x$ is even,
then $x^2 - 6x + 5$ is odd.
Let $x \in \mathbb{Z}$ be arbitrary.
Suppose $x$ is even.
By definition, there exists an integer $k$ such that $x = 2k$.
Then we have
  \begin{align*}
    x^2 - 6x + 5 &= (2k)^2 - 6(2k) + 5 \\
    &= 4k^2 - 12k + 5 \\
    &= 2(2k^2 - 6k + 2) + 1
  \end{align*}
Since the integers are closed under addition, subtraction, and multiplication,
we know $2k^2 - 6x + 2)$ is an integer.
Therefore, we have that that
$x^2 - 6x + 5$ is odd by definition..
\end{proof}

\subsection*{Slide 34}

\textit{Claim:} For integers $n,a,b$, if $n \nmid ab$ then $n \nmid a$ and $n
\nmid b$.

\begin{proof}
  Let $n,a,b \in \mathbb{Z}$ be arbitrary. 
  We instead prove the contrapositive of our claim: 
  if $n \mid a$ or $n \mid b$, then $n \mid ab$.
  We must consider two cases.
  First, suppose $n \mid a$.
  Then, by definition, there exists some $k \in \mathbb{Z}$ 
  where $a = nk$.
  Thus, $ab = nkb$.
  Since the integers are closed under multiplication, $kb$ is an integer and $n \mid ab$ by definition.
  For the second case, suppose instead that $n \mid b$.
  Then, by definition, there exists some $k \in \mathbb{Z}$ 
  where $b = nk$.
  Thus, $ab = nka$.
  Since the integers are closed under multiplication, $ka$ is an integer and $n \mid ab$ by definition.

\end{proof}

\end{document}
