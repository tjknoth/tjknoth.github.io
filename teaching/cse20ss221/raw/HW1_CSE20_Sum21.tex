\documentclass[10pt,letterpaper,unboxed,cm]{article}
\usepackage[margin=1in]{geometry}
\usepackage{graphicx}
\usepackage{enumerate, comment}
\usepackage{adjustbox}

\newcommand{\st}{~\mid~}
\newcommand{\ind}{$~~~~~$}
\usepackage[noend]{algpseudocode}
\usepackage{algorithm,algorithmicx}


\begin{document}


\hfill{CSE 20 Summer 2021}

\hfill{Homework 1}

\hfill{Due date: Sunday, August 8, 2021 at 11:59pm}

\begin{center}
\begin{minipage}[t]{6in}
\rule{\linewidth}{2pt}

{\sc Topics:} Integer representations, circuits, propositional logic
\newline

{\sc References} 
\begin{itemize}
\item Section 4.1: pages 237-240 (up to but not including Modular Arithmetic);
\item Section 4.2: pages 245-253 (up to but not including Modular Exponentiation).
Sections 1.1, 1.2, 1.3 (up to page 31).
\end{itemize}

{\sc Key Concepts} Base expansion, decimal expansion, binary expansion,
hexadecimal expansion, octal expansion, bit shift, 
 BCD,
Proposition, truth value, compound propositions,
negation, conjunction, disjunction, exclusive or,  conditional statements,
converse, contrapositive, inverse, 
biconditional, bitwise operations, consistent specifications, logical equivalence, 
De Morgan's laws.
\newline

\rule{\linewidth}{2pt}
\end{minipage} \hfill

\end{center}



\textbf{In this class, unless the instructions explicitly say otherwise, you are required to justify all your answers.}

 

\begin{enumerate}
 
    
\item ({\bf 16 points})
Color in computer is often represented as a 3-tuple $(R,G,B)$ such that $R,G,B$ are each an integer ranging from 0 to 255.

The $R$ value is the red component, the $G$ value green component and $B$, the blue component. For example: $(0,0,0)$ represents black and $(255,255,255)$ represents white.

A {\bf hex color} is an integer $n$ that has a base 16 \emph{fixed-width} 6 expansion: 
$$n = (r_1r_2g_1g_2b_1b_2)_{16}$$
where $(r_1r_2)_{16}$ is the red component, $(g_1g_2)_{16}$, the green component, and $(b_1b_2)_{16}$, the blue component.


\begin{enumerate}
\item
What is the red, green, and blue values each in base 10 of $(222627)_{16}$
\item
What is the {\bf hex color} that corresponds to the $(R,G,B)$ values of $(104,8,194)$?
 \item
 Convert the (base 10) integer 2,340,902 into hex color. 
\item
{\bf 12-bit color} was used in the 80s on a few different computers. It is an integer $n$ that has a base 16 \emph{fixed-width} {\bf 3} expansion:
$$n=(rgb)_{16}.$$
 Consider the conversion from {\bf hex color} to {\bf 12-bit color}:
 
 Convert a given hex color: $(r_1r_2g_1g_2b_1b_2)_{16}$ to the 12-bit color: $(r_1g_1b_1)_{16}$.
 \begin{enumerate}
 \item
 Convert the hex color $(AB5B93)_{16}$ to 12-bit color.
\item
What is the 12-bit color that corresponds to the (hex color) $(R,G,B)$ values of $(19,202,101)$?
 \end{enumerate}
\end{enumerate}    
    
 \item({\bf 16 points})
We can represent values between 0 and 1 using fractional binary numbers. This is most commonly done by placing the binary \emph{bits} to the right of the \emph{decimal point} (or in this case \emph{binary} point.
A fractional binary number: $(0.b_{-1}b_{-2}\dots b_{-k})_2$ is equal to $b_{-1}/2 + b_{-2}/2^2 + \dots + b_{-k}/2^k$.

For example: 1/2 is written as $(0.1)_2$ in fractional binary.  $5/32$ is written as $(0.00101)_2 = 0/2 + 0/4 + 1/8 + 0/16 + 1/32$.

\newpage

Here is an algorithm for converting a decimal $0<n<1$ into a binary fraction $(0.b_{-1}b_{-2},\dots)_2$:
\begin{enumerate}[1.]
\item
Initialize $v=n$.
\item
Initialize $i = 1$.
\item
Let $p$ be the ``whole number part" of $2v$. (note $p$ can only be 0 or 1).
\item
Set $b_{-i}=p$.
\item
Set $v=2v-p$.
\item
Set $i=i+1$
\item
Go back to step 3.
\end{enumerate}

For Example: to convert 0.375 to binary decimal:
\begin{itemize}
\item
Multiply by 2 to get: 2(0.375) = 0.75
\item
The whole number part is 0 so set $b_{-1}=0$.
\item
Multiply 0.75 by 2: 2(0.75) = 1.5
\item
The whole number part is 1 so set $b_{-2}=1$.
\item
Multiply (the fractional part) 0.5 by 2:
2(0.5) = 1.
\item
The whole number part is 1 so set $b_{-3}=1$.
\item
The fractional part of 1 is 0.0 so all remaining bits will be zero.
\end{itemize}
The result is that $0.375 = 0.011$.
\begin{enumerate}
\item 
Convert the following numbers into fractional binary. (note that there may be infinite repeating sequences of bits so only compute the first 8 bits after the point.)

(Please show your work.)
\begin{itemize}
\item
5/7
\item
1/6
\item
0.61803
\item
0.57721
\end{itemize}
\item
Try some more examples and conjecture as to which fractional binary numbers terminate and which fractional binary numbers have infinite repeating expansions. (not for credit)
\end{enumerate}
    

    \item ({\bf 16 points})
Suppose you are working the door at a stadium. It is your job to count the number of people who enter. You decide to buy a \emph{tally counter} which is a mechanical device that will add 1 to the running total every time you press the lever.

\includegraphics[scale=0.75]{tallycounter_1}

You work at a basketball stadium and you can expect a maximum of 50000 people to come in for a busy event. The tally counter in the picture has 5 wheels each with 10 digits, so it should work out for you since it can count from 00000 to 99999. (You could also use any tally counter with 5 or more wheels. But 4 wheels is not enough.)

The Tally Counter Manufacturer (TCM) allows you to customize your own tally counter with as many wheels as you wish. Each wheel costs \$10. So, in order to save money, you buy a tally counter with 5 wheels which costs a total of \$50.

TCM also makes wheels for their tally counters of sizes (2,3,4,5,6,7,8,9) and charges \$$n$ per wheel with $n$ digits. For example, you could by a tally counter where each wheel has only 5 digits (0,1,2,3,4) and each one of these wheels would cost \$5. Since you know how to read integers in any base, you decide to explore your options.

\begin{enumerate}
\item
Compute the prices of the cheapest possible tally counters with at least 50000 capacity using wheels of sizes:
\begin{center}
2, 3, 4, 5, 6, 7, 8, 9
\end{center} 
where each wheel of size $n$ costs \$$n$.
What wheel size gives you the best price?
\item
Suppose TCM is having a sale and is now charging \$$\sqrt{n}$ per wheel with $n$ digits.

Compute the prices of the cheapest possible tally counters with at least 50000 capacity using wheels of sizes:
\begin{center}
2, 3, 4, 5, 6, 7, 8, 9
\end{center} 
where each wheel of size $n$ costs \$$\sqrt{n}$.
What wheel size gives you the best price?
\item

\end{enumerate}



\item (16 points)
Simple digital devices like calculators, digital clocks, etc. use a display system called 7-segment display. It works by lighting up a combination of the 7 segments for each digit.

\includegraphics[scale=0.2]{7-segment}

Pictured below are the combinations for A,B,C, and D:

\includegraphics[scale=0.2]{7-segment_A}
\includegraphics[scale=0.2]{7-segment_b}
\includegraphics[scale=0.2]{7-segment_C}
\includegraphics[scale=0.2]{7-segment_d}

Let's say that you have a device that only needs to display the letters A, B, C and D.

You can encode these 4 digits with only 2 bits: $x_1,x_0$.

\begin{enumerate}
\item
Fill in the ``truth table" for a circuit that displays the digits 0,1,2,3 in 7-segment display based on the following representation: 00 corresponds to A, 01 corresponds to B, 10 corresponds to C, 11 corresponds to D.

\begin{tabular}{|c|c||c|c|c|c|c|c|c|}
\hline
$x_1$&$x_0$&$~~a~~$&$~~b~~$&$~~c~~$&$~~d~~$&$~~e~~$&$~~f~~$&$~~g~~$\\
\hline
&&&&&&&&\\
$0$&$0$&&&&&&&\\
&&&&&&&&\\
\hline
&&&&&&&&\\
$0$&$1$&&&&&&&\\
&&&&&&&&\\
\hline
&&&&&&&&\\
$1$&$0$&&&&&&&\\
&&&&&&&&\\
\hline
&&&&&&&&\\
$1$&$1$&&&&&&&\\
&&&&&&&&\\
\hline
\end{tabular}

\item
Draw the circuit.
\emph{(you may rearrange your outputs circuits and duplicate your inputs to make your circuit neater. )}
\end{enumerate}

   
    

   
    
    \item (16 points) Group the following 8 compound propositions into 4 pairs of logically equivalent propositions.
    
   
    
    \begin{enumerate}
        \item $(q\wedge r)\vee \lnot p$
        \item $(p\leftrightarrow r)\to (p\leftrightarrow q)$
        \item $(p \to q)\vee (p \to r)$
         \item $\lnot r \vee ((p \to \lnot q) \wedge (p\vee q))$
         \item $((p\oplus q)\wedge (q\oplus r))\to(p\oplus r)$
         \item $p \to (q\vee r)$
          \item $r\to (p\oplus q)$
          \item $(p \to q)\wedge (p \to r)$
          
          
    \end{enumerate}
    

    


\item ( 16 points)
The dog, the cat, the pig and the turtle are all accused of a crime. The police have deduced the following:

\begin{enumerate}[A:]
\item
If the dog is innocent then the cat is innocent and the pig is innocent.
\item
If the pig is innocent then the cat is innocent or the dog is innocent.
\item
If the turtle is guilty then the cat is guilty and the pig is innocent.
\end{enumerate}

\begin{enumerate}
\item
Convert the system of statements into compound propositions.
\item
Is this set (A,B,C) of statements consistent? Why or why not. Is there enough information to make a conclusion?
\item
Imagine that a piece of evidence was discovered later that said D: ``if the turtle is innocent then the cat is guilty." If we add this statement to the set, is this set of statements (A,B,C,D) consistent? If it is consistent, can you make a conclusion?
\item
Imagine that a piece of evidence was discovered later that said E: ``if the turtle is innocent or the dog is innocent then the cat is guilty and the pig is innocent." If we add this statement to the set, is this set of statements (A,B,C,E) consistent? If it is consistent, can you make a conclusion?
\end{enumerate}

\end{enumerate}


\end{document}
